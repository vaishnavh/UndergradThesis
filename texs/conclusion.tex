\chapter{CONCLUSION AND FUTURE WORK} \label{ch_concl}
%Chapter 6
%%%%%%%%%%%%%%%%%%%%%%%%%%%%%%%%%%%%%%%%%%%%%%%%%%%%%%%%%%%%%%%%%%%%%%%%%%%%%%%%%%%%%%%%%%%%%%%%%%%%%%%%%%%%%%%%%%%%%%%%%
\section{Conclusion}
 In this work, we have designed distance oracle for chordal graphs. We used the structural properties of underlying clique tree
 structure of chordal graphs to build a set of data structures that can answer distance queries in constant time.
 We proved that the hitting set for the path joining cliques $C_i$ and $C_j$ encodes the distance between the pair of vertices present in 
 $C_i$ and $C_j$ respectively. Lemma \ref{hitsel} and Theorem \ref{the:4} represent an interesting connection between the hitting set for a path of cliques
 in clique tree and the distance between the vertices present in the end cliques of the path. We also proved that by keeping the sizes of 
 the hitting sets of cliques on all paths originating from a single vertex in the clique tree, we can compute hitting set for the path joining 
 any pair of cliques in constant time. This observation is important to answer queries in constant time.
 We essentially proved that a linear size distance oracle for chordal graphs can be built in $O(n^2)$ time and $O(n^2)$ space. 
 Subsequent queries for the distance between any pair of vertices of the graph can be answered approximately in constant time. 
 Finally, by maintaining the approximate hitting sets as well, 
 along with the whole clique tree, we can modify the distance oracle in this thesis to use $O(n^2)$ space, 
 and  respond to distance queries with paths of length at most $2d_G(u,v)+8$.
 \section{Future Work}
 A tree $t$-spanner of a graph can be visualized as a distance oracle of the graph that can answer distance queries with stretch of at most $t$.
 We know that the hitting set encodes the distance information of the graph.
 Therefore, the idea of hitting set can be further extended to compute a tree t-spanner for chordal graphs. 
 The other area of work is to use the idea of hitting set to compute $G^2$ matrix for split graphs without using matrix multiplication.
 Due to a reduction by Han et al.~\cite{han}, we know that if $G^2$ matrix of a split graph can be computed efficiently then 
 it can be efficiently computed for general graphs as well.
 Therefore, by using this procedure we can remove the dependency of solving $\textsc{APSP}$ for unweighted general graphs~\cite{seidel} on matrix multiplication. 
 Hence, we can reduce the time complexity to compute $\textsc{APSP}$ matrix for general unweighted graph.
%%%%%%%%%%%%%%%%%%%%%%%%%%%%%%%%%%%%%%%%%%%%%%%%%%%%%%%%%%%%%%%%%%%%%%%%%%%%%%%%%%%%%%%%%%%%%%%%%%%%%%%%%%%%%%%%%%%%%%%%
